\documentclass[a4paper,14pt, unknownkeysallowed]{extreport}

\usepackage{cmap} % Улучшенный поиск русских слов в полученном pdf-файле
\usepackage[T2A]{fontenc} % Поддержка русских букв
\usepackage[utf8]{inputenc} % Кодировка utf8
\usepackage[english,russian]{babel} % Языки: русский, английский
\usepackage{enumitem}


\usepackage{threeparttable}

\usepackage[14pt]{extsizes}

\usepackage{caption}
\captionsetup{labelsep=endash}
\captionsetup[figure]{name={Рисунок}}

% \usepackage{ctable}
% \captionsetup[table]{justification=raggedleft,singlelinecheck=off}

\usepackage{amsmath}

\usepackage{geometry}
\geometry{left=30mm}
\geometry{right=10mm}
\geometry{top=20mm}
\geometry{bottom=20mm}

\usepackage{titlesec}
\titleformat{\section}
	{\normalsize\bfseries}
	{\thesection}
	{1em}{}
\titlespacing*{\chapter}{0pt}{-30pt}{8pt}
\titlespacing*{\section}{\parindent}{*4}{*4}
\titlespacing*{\subsection}{\parindent}{*4}{*4}

\usepackage{setspace}
\onehalfspacing % Полуторный интервал

\frenchspacing
\usepackage{indentfirst} % Красная строка

\usepackage{titlesec}
\titleformat{\chapter}{\LARGE\bfseries}{\thechapter}{20pt}{\LARGE\bfseries}
\titleformat{\section}{\Large\bfseries}{\thesection}{20pt}{\Large\bfseries}

\usepackage{multirow}
\usepackage{listings}
\usepackage{xcolor}

% Для листинга кода:
\lstset{%
	language=python,   					% выбор языка для подсветки	
	basicstyle=\small\sffamily,			% размер и начертание шрифта для подсветки кода
	numbers=left,						% где поставить нумерацию строк (слева\справа)
	%numberstyle=,						% размер шрифта для номеров строк
	stepnumber=1,						% размер шага между двумя номерами строк
	numbersep=5pt,						% как далеко отстоят номера строк от подсвечиваемого кода
	frame=single,						% рисовать рамку вокруг кода
	tabsize=4,							% размер табуляции по умолчанию равен 4 пробелам
	captionpos=t,						% позиция заголовка вверху [t] или внизу [b]
	breaklines=true,					
	breakatwhitespace=true,				% переносить строки только если есть пробел
	escapeinside={\#*}{*)},				% если нужно добавить комментарии в коде
	backgroundcolor=\color{white},
}


\usepackage{pgfplots}
\usetikzlibrary{datavisualization}
\usetikzlibrary{datavisualization.formats.functions}

\usepackage{graphicx}
\newcommand{\img}[3] {
	\begin{figure}[h!]
		\center{\includegraphics[height=#1]{img/#2}}
		\caption{#3}
		\label{img:#2}
	\end{figure}
}


\usepackage[justification=centering]{caption} % Настройка подписей float объектов

\usepackage[unicode,pdftex]{hyperref} % Ссылки в pdf
\hypersetup{hidelinks}

\usepackage{csvsimple}

\newcommand{\code}[1]{\texttt{#1}}





\begin{document}


\begin{titlepage}
	\newgeometry{pdftex, left=2cm, right=2cm, top=2.5cm, bottom=2.5cm}
	\fontsize{12pt}{12pt}\selectfont
	\noindent \begin{minipage}{0.15\textwidth}
		\includegraphics[width=\linewidth]{img/b_logo.jpg}
	\end{minipage}
	\noindent\begin{minipage}{0.9\textwidth}\centering
		\textbf{Министерство науки и высшего образования Российской Федерации}\\
		\textbf{Федеральное государственное бюджетное образовательное учреждение высшего образования}\\
		\textbf{«Московский государственный технический университет имени Н. Э.~Баумана}\\
		\textbf{(национальный исследовательский университет)»}\\
		\textbf{(МГТУ им. Н. Э.~Баумана)}
	\end{minipage}
	
	\noindent\rule{18cm}{3pt}
	\newline\newline
	\noindent ФАКУЛЬТЕТ $\underline{\text{«Информатика и системы управления»~~~~~~~~~~~~~~~~~~~~~~~~~~~~~~~~~~~~~~~~~~~~~~~~~~~~~~~}}$ \newline\newline
	\noindent КАФЕДРА $\underline{\text{«Программное обеспечение ЭВМ и информационные технологии»~~~~~~~~~~~~~~~~~~~~~~~}}$\newline\newline\newline\newline\newline\newline\newline
	
	
	\begin{center}
		\noindent\begin{minipage}{1.3\textwidth}\centering
		\Large\textbf{   ~~~ Домашняя работа №1}\newline
		\textbf{по дисциплине "Анализ Алгоритмов"}\newline\newline\newline
		\end{minipage}
	\end{center}
	
	\noindent\textbf{Тема} 			$\underline{\text{Графовые представления}}$\newline\newline
	\noindent\textbf{Студент} 		$\underline{\text{Ковалец К. Э.}}$\newline\newline
	\noindent\textbf{Группа} 		$\underline{\text{ИУ7-53Б}}$\newline\newline
	\noindent\textbf{Преподаватель} $\underline{\text{Волкова Л. Л.}}$\newline
	
	\begin{center}
		\vfill
		Москва~---~\the\year
		~г.
	\end{center}
	\restoregeometry
\end{titlepage}





\chapter{Листинг кода}

\begin{center}
\captionsetup{justification=raggedright,singlelinecheck=off}
\begin{lstlisting}[label=lst:classical_alg,caption=Функция нахождения расстояния Левенштейна итеративно]
def levenstein_table():

    s1 = input("\ninput 1 string: ")       			# 1
    s2 = input("input 2 string: ")           		# 2

    len1 = len(s1)                                  # 3
    len2 = len(s2)                                  # 4

    M = [[0] * (len2 + 1) for _ in range(len1 + 1)] # 5

    for i in range(len1 + 1):                       # 6    
        M[i][0] = i                                 # 7
    
    for j in range(len2 + 1):                       # 8   
        M[0][j] = j                                 # 9   

    for i in range(1, len1 + 1):                    # 10   
        for j in range(1, len2 + 1):                # 11
                              
            A = M[i - 1][j    ] + 1                 # 12
            D = M[i    ][j - 1] + 1                 # 13
            C = M[i - 1][j - 1]                     # 14
            
            if s1[i - 1] != s2[j - 1]:              # 15
                C += 1                              # 16

            M[i][j] = min(A, D, C)                  # 17

    return M[-1][-1]
\end{lstlisting}
\end{center}

\clearpage

\chapter{Графовые представления}

На рис. \ref{fig:oper_graph} - \ref{fig:inf_his} приведены графы (операционный, информационый, операционной истории программы, информационый истории программы) для матричной реализации алгоритма Левенштейна.

\begin{figure}[h]
	\centering
	\includegraphics[scale=0.6]{img/oper_graph.png}
	\caption{Операционный граф программы}
	\label{fig:oper_graph}
\end{figure}

\clearpage

\begin{figure}[h]
	\centering
	\includegraphics[scale=0.5]{img/inf_graph.png}
	\caption{Информационный граф программы}
	\label{fig:inf_graph}
\end{figure}

\clearpage

\begin{figure}[h]
	\centering
	\includegraphics[scale=0.22]{img/oper_history_graph.png}
	\caption{Граф операционной истории программы}
	\label{fig:oper_his}
\end{figure}

\clearpage

\begin{figure}[h]
	\centering
	\includegraphics[scale=0.13]{img/inf_history_graph.png}
	\caption{Граф информационной истории программы}
	\label{fig:inf_his}
\end{figure}

\clearpage

\end{document}
